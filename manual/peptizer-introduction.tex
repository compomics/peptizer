\chapter{Introduction}
\npar Large datasets generated by \textbf{MS/MS-driven proteomics} are most commonly analyzed by algorithms such as \textbf{MASCOT}, X!Tandem and SEQUEST. These perform a comprehensive sequence database search and thereby suggesting multiple peptide hypothesises for each MS/MS spectrum. Typically, the highest scoring peptide hypothesis with an e-value lower then 0.05 is accepted. However, the difficulty is to verify \textbf{whether that suggested peptide hypothesis is correct}. By post-processing database search results, an \textbf{extra layer of validation} can be added to these peptide hypothesises. \textbf{Peptizer} was developed as a \textbf{configurable post-processing }platform that can be applied to \textbf{separate suspicious peptide hypothesises} from valid ones, and subjecting the former to manual validation.
\npar Instead of verifying peptide hypothesises using fixed assumptions, Peptizer makes use of so-called \textbf{pluggable assumptions}. These are called \textbf{Agents} and can vote on a specific property of a peptide hypothesis. By combining a group of Agents, their votes are aggregated and all together form a profile that separates peptide hypothesises in .
\npar Peptizer presents the selected peptide identifications in an \textbf{extensive manual validation environment}. Therein, general and specific information, together with spectrum-derived information place a critical scientist in an information-rich and therefore \textbf{optimal position to validate }database search results.
