\chapter{\label{installation}Installing Peptizer}
\urlstyle{leostyle}
\npar
Installing Peptizer is a two-step process: first, download and install Java version 1.5 or higher and second, download the Peptizer files from the project website at \url{http://genesis.ugent.be/peptizer}.
\section{Downloading Java}
As Peptizer is a Java application, Java must be installed properly. Check this by opening the console
%
\begin{center}
Windows   -   \textit{Windows Start// Run // cmd}\\
Unix   -   \textit{Open the shell of the distribution}
\end{center}
%
and type
%
\begin{center}
java -version
\end{center}
%
Now something like
%
\begin{center}
java version "1.6.0\_03"
\end{center}
%
should appear which indicates that Java version 1.6 is installed properly. If Java version 1.5 or later is not installed, then go to \url{http://java.com/} and follow the instructions to install the latest Java version.\\
\section{Downloading the Peptizer files}
There are three types of Peptizer files:
\begin{enumerate}
	\item Java libraries
	\item Configuration files 
	\item Start-up script
\end{enumerate}
%
\npar You can install these files properly by the Java installer you find at the project website (\url{http://genesis.ugent.be/peptizer/download/installer.html}).\\ Start Peptizer through \textit{double clicking }the start-up script.
\npar 

